\section{Prototyp algorytmu}
Po zapoznaniu z wiadomościami teoretycznymi i ustaleniu zakresu projektu, przystąpiono do zaprojektowania prototypów algorytmów $kNN$ oraz $eNN$ w środowisku Matlab.

Algorytm badany był przy użyciu zbioru sygnałów $EKG$ zredukowanych do opisu cech zespołów $QRS$, uporządkowanych do zbioru trzech klas. \textbf{JAKICH!}

Dane wejściowe poddano normalizacji, która opisana jest równaniem \ref{eq:normalize-data}.

\begin{equation}
\label{eq:normalize-data}
x_j = \frac{x_j - \mu(x_j)}{\sigma(x_j)}, j=1,2...N
\end{equation}
gdzie $x_j$ opisuje $j$-tą kolumnę macierzy danych.

Skuteczność badano niezależnie dla każdego pliku. Wyniki przedstawiono w tabeli \ref{tab:matlab-skutecznosc}.

\begin{table}[!htb]
	\centering
	\begin{tabular}{|c|r|r|r|r|r|r|r|r|r|}
		\hline
		& \multicolumn{4}{c|}{$kNN$} & \multicolumn{4}{|c|}{$eNN$} \\
		\hline
		$Plik$ & 1 & 2 & 3 & $\Sigma$ & 1 & 2 & 3 & $\Sigma$ \\
		\hline
	\end{tabular}
	\caption{Wyniki pracy algorytmu kNN}
	\label{tab:matlab-skutecznosc}
	

\end{table}
+ enn + czas wykonania